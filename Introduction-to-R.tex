% Options for packages loaded elsewhere
\PassOptionsToPackage{unicode}{hyperref}
\PassOptionsToPackage{hyphens}{url}
%
\documentclass[
]{article}
\usepackage{amsmath,amssymb}
\usepackage{lmodern}
\usepackage{iftex}
\ifPDFTeX
  \usepackage[T1]{fontenc}
  \usepackage[utf8]{inputenc}
  \usepackage{textcomp} % provide euro and other symbols
\else % if luatex or xetex
  \usepackage{unicode-math}
  \defaultfontfeatures{Scale=MatchLowercase}
  \defaultfontfeatures[\rmfamily]{Ligatures=TeX,Scale=1}
\fi
% Use upquote if available, for straight quotes in verbatim environments
\IfFileExists{upquote.sty}{\usepackage{upquote}}{}
\IfFileExists{microtype.sty}{% use microtype if available
  \usepackage[]{microtype}
  \UseMicrotypeSet[protrusion]{basicmath} % disable protrusion for tt fonts
}{}
\makeatletter
\@ifundefined{KOMAClassName}{% if non-KOMA class
  \IfFileExists{parskip.sty}{%
    \usepackage{parskip}
  }{% else
    \setlength{\parindent}{0pt}
    \setlength{\parskip}{6pt plus 2pt minus 1pt}}
}{% if KOMA class
  \KOMAoptions{parskip=half}}
\makeatother
\usepackage{xcolor}
\usepackage[margin=1in]{geometry}
\usepackage{color}
\usepackage{fancyvrb}
\newcommand{\VerbBar}{|}
\newcommand{\VERB}{\Verb[commandchars=\\\{\}]}
\DefineVerbatimEnvironment{Highlighting}{Verbatim}{commandchars=\\\{\}}
% Add ',fontsize=\small' for more characters per line
\usepackage{framed}
\definecolor{shadecolor}{RGB}{248,248,248}
\newenvironment{Shaded}{\begin{snugshade}}{\end{snugshade}}
\newcommand{\AlertTok}[1]{\textcolor[rgb]{0.94,0.16,0.16}{#1}}
\newcommand{\AnnotationTok}[1]{\textcolor[rgb]{0.56,0.35,0.01}{\textbf{\textit{#1}}}}
\newcommand{\AttributeTok}[1]{\textcolor[rgb]{0.77,0.63,0.00}{#1}}
\newcommand{\BaseNTok}[1]{\textcolor[rgb]{0.00,0.00,0.81}{#1}}
\newcommand{\BuiltInTok}[1]{#1}
\newcommand{\CharTok}[1]{\textcolor[rgb]{0.31,0.60,0.02}{#1}}
\newcommand{\CommentTok}[1]{\textcolor[rgb]{0.56,0.35,0.01}{\textit{#1}}}
\newcommand{\CommentVarTok}[1]{\textcolor[rgb]{0.56,0.35,0.01}{\textbf{\textit{#1}}}}
\newcommand{\ConstantTok}[1]{\textcolor[rgb]{0.00,0.00,0.00}{#1}}
\newcommand{\ControlFlowTok}[1]{\textcolor[rgb]{0.13,0.29,0.53}{\textbf{#1}}}
\newcommand{\DataTypeTok}[1]{\textcolor[rgb]{0.13,0.29,0.53}{#1}}
\newcommand{\DecValTok}[1]{\textcolor[rgb]{0.00,0.00,0.81}{#1}}
\newcommand{\DocumentationTok}[1]{\textcolor[rgb]{0.56,0.35,0.01}{\textbf{\textit{#1}}}}
\newcommand{\ErrorTok}[1]{\textcolor[rgb]{0.64,0.00,0.00}{\textbf{#1}}}
\newcommand{\ExtensionTok}[1]{#1}
\newcommand{\FloatTok}[1]{\textcolor[rgb]{0.00,0.00,0.81}{#1}}
\newcommand{\FunctionTok}[1]{\textcolor[rgb]{0.00,0.00,0.00}{#1}}
\newcommand{\ImportTok}[1]{#1}
\newcommand{\InformationTok}[1]{\textcolor[rgb]{0.56,0.35,0.01}{\textbf{\textit{#1}}}}
\newcommand{\KeywordTok}[1]{\textcolor[rgb]{0.13,0.29,0.53}{\textbf{#1}}}
\newcommand{\NormalTok}[1]{#1}
\newcommand{\OperatorTok}[1]{\textcolor[rgb]{0.81,0.36,0.00}{\textbf{#1}}}
\newcommand{\OtherTok}[1]{\textcolor[rgb]{0.56,0.35,0.01}{#1}}
\newcommand{\PreprocessorTok}[1]{\textcolor[rgb]{0.56,0.35,0.01}{\textit{#1}}}
\newcommand{\RegionMarkerTok}[1]{#1}
\newcommand{\SpecialCharTok}[1]{\textcolor[rgb]{0.00,0.00,0.00}{#1}}
\newcommand{\SpecialStringTok}[1]{\textcolor[rgb]{0.31,0.60,0.02}{#1}}
\newcommand{\StringTok}[1]{\textcolor[rgb]{0.31,0.60,0.02}{#1}}
\newcommand{\VariableTok}[1]{\textcolor[rgb]{0.00,0.00,0.00}{#1}}
\newcommand{\VerbatimStringTok}[1]{\textcolor[rgb]{0.31,0.60,0.02}{#1}}
\newcommand{\WarningTok}[1]{\textcolor[rgb]{0.56,0.35,0.01}{\textbf{\textit{#1}}}}
\usepackage{graphicx}
\makeatletter
\def\maxwidth{\ifdim\Gin@nat@width>\linewidth\linewidth\else\Gin@nat@width\fi}
\def\maxheight{\ifdim\Gin@nat@height>\textheight\textheight\else\Gin@nat@height\fi}
\makeatother
% Scale images if necessary, so that they will not overflow the page
% margins by default, and it is still possible to overwrite the defaults
% using explicit options in \includegraphics[width, height, ...]{}
\setkeys{Gin}{width=\maxwidth,height=\maxheight,keepaspectratio}
% Set default figure placement to htbp
\makeatletter
\def\fps@figure{htbp}
\makeatother
\setlength{\emergencystretch}{3em} % prevent overfull lines
\providecommand{\tightlist}{%
  \setlength{\itemsep}{0pt}\setlength{\parskip}{0pt}}
\setcounter{secnumdepth}{-\maxdimen} % remove section numbering
\ifLuaTeX
  \usepackage{selnolig}  % disable illegal ligatures
\fi
\IfFileExists{bookmark.sty}{\usepackage{bookmark}}{\usepackage{hyperref}}
\IfFileExists{xurl.sty}{\usepackage{xurl}}{} % add URL line breaks if available
\urlstyle{same} % disable monospaced font for URLs
\hypersetup{
  pdftitle={Introduction to R Programming},
  pdfauthor={AHMAD RAZA},
  hidelinks,
  pdfcreator={LaTeX via pandoc}}

\title{Introduction to R Programming}
\author{AHMAD RAZA}
\date{September 14, 2022}

\begin{document}
\maketitle

\hypertarget{note}{%
\subparagraph{NOTE:}\label{note}}

\begin{enumerate}
\def\labelenumi{\arabic{enumi}.}
\tightlist
\item
  Change author name and date to your exercise submission date in above
  section
\item
  Your code MUST execute without any errors.
\item
  You can add more lines in your code as required.
\end{enumerate}

\hypertarget{section-1-data-types-and-operations-pt.-1}{%
\subsection{Section 1: Data Types and Operations Pt.
1}\label{section-1-data-types-and-operations-pt.-1}}

\hypertarget{question-1}{%
\subsubsection{Question 1}\label{question-1}}

\textbf{Create the variables with the following composition:}\\
1. A vector containing each letter of your first name as its elements.\\
2. A variable that contains your name concatenated from the vector
created in (1)\\
3. A variable containing a sequence from 100 to 120.\\
4. Create a matrix of 3x3 dimensions that contains the even sequence of
numbers starting from 2.\\
5. Assign names to the variables.

\begin{Shaded}
\begin{Highlighting}[]
\NormalTok{vector1 }\OtherTok{\textless{}{-}} \FunctionTok{c}\NormalTok{(}\StringTok{"a"}\NormalTok{,}\StringTok{"h"}\NormalTok{,}\StringTok{"m"}\NormalTok{,}\StringTok{"a"}\NormalTok{,}\StringTok{"d"}\NormalTok{)}
\FunctionTok{print}\NormalTok{(vector1)}
\end{Highlighting}
\end{Shaded}

\begin{verbatim}
## [1] "a" "h" "m" "a" "d"
\end{verbatim}

\begin{Shaded}
\begin{Highlighting}[]
\NormalTok{name }\OtherTok{=} \FunctionTok{paste}\NormalTok{(vector1,}\AttributeTok{collapse =} \StringTok{""}\NormalTok{)}
\FunctionTok{print}\NormalTok{(name)}
\end{Highlighting}
\end{Shaded}

\begin{verbatim}
## [1] "ahmad"
\end{verbatim}

\begin{Shaded}
\begin{Highlighting}[]
\NormalTok{var1 }\OtherTok{\textless{}{-}} \DecValTok{100}\SpecialCharTok{:}\DecValTok{120}
\FunctionTok{print}\NormalTok{(var1)}
\end{Highlighting}
\end{Shaded}

\begin{verbatim}
##  [1] 100 101 102 103 104 105 106 107 108 109 110 111 112 113 114 115 116 117 118
## [20] 119 120
\end{verbatim}

\begin{Shaded}
\begin{Highlighting}[]
\NormalTok{numbers }\OtherTok{\textless{}{-}} \FunctionTok{seq}\NormalTok{(}\AttributeTok{from =} \DecValTok{2}\NormalTok{, }\AttributeTok{to =} \DecValTok{18}\NormalTok{, }\AttributeTok{by =} \DecValTok{2}\NormalTok{)}
\NormalTok{mat1 }\OtherTok{=} \FunctionTok{matrix}\NormalTok{(numbers,}\AttributeTok{nrow =} \DecValTok{3}\NormalTok{,}\AttributeTok{ncol=}\DecValTok{3}\NormalTok{)}
\FunctionTok{print}\NormalTok{(mat1)}
\end{Highlighting}
\end{Shaded}

\begin{verbatim}
##      [,1] [,2] [,3]
## [1,]    2    8   14
## [2,]    4   10   16
## [3,]    6   12   18
\end{verbatim}

\begin{Shaded}
\begin{Highlighting}[]
\FunctionTok{row.names}\NormalTok{(mat1) }\OtherTok{\textless{}{-}} \FunctionTok{c}\NormalTok{(}\StringTok{"r1"}\NormalTok{,}\StringTok{"r2"}\NormalTok{,}\StringTok{"r3"}\NormalTok{)}
\FunctionTok{colnames}\NormalTok{(mat1) }\OtherTok{\textless{}{-}} \FunctionTok{c}\NormalTok{(}\StringTok{"c1"}\NormalTok{,}\StringTok{"c2"}\NormalTok{,}\StringTok{"c3"}\NormalTok{)}
\FunctionTok{print}\NormalTok{(mat1)}
\end{Highlighting}
\end{Shaded}

\begin{verbatim}
##    c1 c2 c3
## r1  2  8 14
## r2  4 10 16
## r3  6 12 18
\end{verbatim}

\begin{Shaded}
\begin{Highlighting}[]
\DocumentationTok{\#\#\#\# End solution \#\#\#\#}
\end{Highlighting}
\end{Shaded}

\hypertarget{question-2}{%
\subsubsection{Question 2}\label{question-2}}

\textbf{Create a factor variable emp\_status:}\\
1. Containing the categorical variables: Employed, Unemployed,
Self-Employed, with each level appearing atleast more than 2.\\
2. Display the levels and the factor variable as a table.\\
3. Unclass the elements of the factor variable.

\begin{Shaded}
\begin{Highlighting}[]
\DocumentationTok{\#\#\#\# Start solution \#\#\#\#}
\end{Highlighting}
\end{Shaded}

\begin{Shaded}
\begin{Highlighting}[]
\NormalTok{emp\_status }\OtherTok{\textless{}{-}} \FunctionTok{factor}\NormalTok{(}\FunctionTok{c}\NormalTok{(}\StringTok{"Employed"}\NormalTok{,}\StringTok{"Employed"}\NormalTok{,}\StringTok{"Unemployed"}\NormalTok{,}\StringTok{"Self{-}Employed"}\NormalTok{,}\StringTok{"Unemployed"}\NormalTok{,}\StringTok{"Employed"}\NormalTok{,}\StringTok{"Unemployed"}\NormalTok{,}\StringTok{"Self{-}Employed"}\NormalTok{,}\StringTok{"Self{-}Employed"}\NormalTok{),}\AttributeTok{levels =} \FunctionTok{c}\NormalTok{(}\StringTok{"Employed"}\NormalTok{,}\StringTok{"Unemployed"}\NormalTok{,}\StringTok{"Self{-}Employed"}\NormalTok{))}
\FunctionTok{print}\NormalTok{(emp\_status)}
\end{Highlighting}
\end{Shaded}

\begin{verbatim}
## [1] Employed      Employed      Unemployed    Self-Employed Unemployed   
## [6] Employed      Unemployed    Self-Employed Self-Employed
## Levels: Employed Unemployed Self-Employed
\end{verbatim}

\begin{Shaded}
\begin{Highlighting}[]
\FunctionTok{table}\NormalTok{(emp\_status)}
\end{Highlighting}
\end{Shaded}

\begin{verbatim}
## emp_status
##      Employed    Unemployed Self-Employed 
##             3             3             3
\end{verbatim}

\begin{Shaded}
\begin{Highlighting}[]
\FunctionTok{unclass}\NormalTok{(emp\_status)}
\end{Highlighting}
\end{Shaded}

\begin{verbatim}
## [1] 1 1 2 3 2 1 2 3 3
## attr(,"levels")
## [1] "Employed"      "Unemployed"    "Self-Employed"
\end{verbatim}

\begin{Shaded}
\begin{Highlighting}[]
\DocumentationTok{\#\#\#\# End solution \#\#\#\#}
\end{Highlighting}
\end{Shaded}

\hypertarget{question-3}{%
\subsubsection{Question 3}\label{question-3}}

\textbf{Create a dataframe object called bank\_customers:}\\
1. the data frame will have three columns: CustomerID, hasAccount,
totalBalance\\
2. Fill the data as follows\\
a. Alicia does not have an account. She is here for inquiry\\
b. Nancy is here to check on her account balance of USD 10,000.\\
c.~Fernando is here to deposit USD 100 in his account which had no
balance.\\
d.~Louis will withdraw all his money from the account that had USD
2,000.\\
e. Diane is here for information.\\
3. For customers that do not have a value, use NA as placeholder.\\
4. Print the number of rows, columns and names for the data frame.

\begin{Shaded}
\begin{Highlighting}[]
\DocumentationTok{\#\#\#\# Start solution \#\#\#\#}
\end{Highlighting}
\end{Shaded}

\begin{Shaded}
\begin{Highlighting}[]
\NormalTok{bank\_customers }\OtherTok{\textless{}{-}} \FunctionTok{data.frame}\NormalTok{(}\AttributeTok{CustomerID=}\DecValTok{101}\SpecialCharTok{:}\DecValTok{105}\NormalTok{,}\AttributeTok{hasAccount=}\FunctionTok{c}\NormalTok{(}\ConstantTok{FALSE}\NormalTok{,}\ConstantTok{TRUE}\NormalTok{,}\ConstantTok{TRUE}\NormalTok{,}\ConstantTok{TRUE}\NormalTok{,}\ConstantTok{NA}\NormalTok{),}\AttributeTok{totalBalance=} \FunctionTok{c}\NormalTok{(}\ConstantTok{NA}\NormalTok{,}\DecValTok{10000}\NormalTok{,}\DecValTok{0}\NormalTok{,}\DecValTok{2000}\NormalTok{,}\ConstantTok{NA}\NormalTok{))}

\FunctionTok{row.names}\NormalTok{(bank\_customers)}\OtherTok{\textless{}{-}}\FunctionTok{c}\NormalTok{(}\StringTok{"Alicia"}\NormalTok{,}\StringTok{"Nancy"}\NormalTok{,}\StringTok{"Fernando"}\NormalTok{,}\StringTok{"Louis"}\NormalTok{,}\StringTok{"Diane"}\NormalTok{)}
\FunctionTok{print}\NormalTok{(bank\_customers)}
\end{Highlighting}
\end{Shaded}

\begin{verbatim}
##          CustomerID hasAccount totalBalance
## Alicia          101      FALSE           NA
## Nancy           102       TRUE        10000
## Fernando        103       TRUE            0
## Louis           104       TRUE         2000
## Diane           105         NA           NA
\end{verbatim}

\begin{Shaded}
\begin{Highlighting}[]
\CommentTok{\#new record added}
\CommentTok{\#newEmployee \textless{}{-} data.frame(CustomerID=106,hasAccount=TRUE,totalBalance=500)}
\CommentTok{\#rbind(bank\_customers,newEmployee)}
\end{Highlighting}
\end{Shaded}

\begin{Shaded}
\begin{Highlighting}[]
\FunctionTok{nrow}\NormalTok{(bank\_customers)}
\end{Highlighting}
\end{Shaded}

\begin{verbatim}
## [1] 5
\end{verbatim}

\begin{Shaded}
\begin{Highlighting}[]
\FunctionTok{ncol}\NormalTok{(bank\_customers)}
\end{Highlighting}
\end{Shaded}

\begin{verbatim}
## [1] 3
\end{verbatim}

\begin{Shaded}
\begin{Highlighting}[]
\FunctionTok{dimnames}\NormalTok{(bank\_customers)}
\end{Highlighting}
\end{Shaded}

\begin{verbatim}
## [[1]]
## [1] "Alicia"   "Nancy"    "Fernando" "Louis"    "Diane"   
## 
## [[2]]
## [1] "CustomerID"   "hasAccount"   "totalBalance"
\end{verbatim}

\begin{Shaded}
\begin{Highlighting}[]
\DocumentationTok{\#\#\#\# End solution \#\#\#\#}
\end{Highlighting}
\end{Shaded}

\hypertarget{good-job-you-have-successfully-completed-the-section}{%
\subsubsection{Good Job! You have successfully completed the
section!}\label{good-job-you-have-successfully-completed-the-section}}

\hypertarget{section-2-control-structures}{%
\subsection{Section 2: Control
Structures}\label{section-2-control-structures}}

\hypertarget{question-1-1}{%
\subsubsection{Question 1}\label{question-1-1}}

\textbf{Create a variable containing a sequence of numbers from 1 to
100:}\\
1. Iterate over the variables and print those numbers which are prime.

\textbf{Create a variable var with a value of 1:}\\
1. Create a while loop and increase the value of var at each
iteration.\\
2. When you reach the 10th prime number, terminate the loop and print
the number.

\begin{Shaded}
\begin{Highlighting}[]
\DocumentationTok{\#\#\#\# Start solution \#\#\#\#}
\end{Highlighting}
\end{Shaded}

\begin{Shaded}
\begin{Highlighting}[]
\CommentTok{\# Question 1(a)}
\NormalTok{getPrimeNumbers }\OtherTok{\textless{}{-}} \ControlFlowTok{function}\NormalTok{(var2)\{}
     
\NormalTok{   primeMarker}\OtherTok{=} \ConstantTok{TRUE}  
\NormalTok{   var3 }\OtherTok{\textless{}{-}} \FunctionTok{c}\NormalTok{()}
   \ControlFlowTok{for}\NormalTok{ (x }\ControlFlowTok{in}\NormalTok{ var2)\{}
     \ControlFlowTok{if}\NormalTok{(x }\SpecialCharTok{==} \DecValTok{1}\NormalTok{)\{}
\NormalTok{          var3 }\OtherTok{\textless{}{-}} \FunctionTok{c}\NormalTok{(var3,x)}
          \ControlFlowTok{next}
\NormalTok{     \}}
     \ControlFlowTok{for}\NormalTok{ (y }\ControlFlowTok{in}\NormalTok{ var2)\{}
          \ControlFlowTok{if}\NormalTok{(y }\SpecialCharTok{==} \DecValTok{1}\NormalTok{)}\ControlFlowTok{next}
          \ControlFlowTok{else} \ControlFlowTok{if}\NormalTok{(x }\SpecialCharTok{==}\NormalTok{ y)}\ControlFlowTok{next}
          
          \ControlFlowTok{if}\NormalTok{ ( x }\SpecialCharTok{\%\%}\NormalTok{ y }\SpecialCharTok{==} \DecValTok{0}\NormalTok{)\{}
\NormalTok{               primeMarker }\OtherTok{=} \ConstantTok{FALSE}
               \ControlFlowTok{break}\NormalTok{\}}
          \ControlFlowTok{else}\NormalTok{ \{}
\NormalTok{               primeMarker}\OtherTok{=} \ConstantTok{TRUE}
               
\NormalTok{          \}}
          
\NormalTok{     \} }
     
     \ControlFlowTok{if}\NormalTok{(primeMarker }\SpecialCharTok{==} \ConstantTok{TRUE}\NormalTok{)\{}
          \CommentTok{\# vector concatenation}
\NormalTok{          var3 }\OtherTok{\textless{}{-}} \FunctionTok{c}\NormalTok{(var3,x)}
\NormalTok{          primeMarker }\SpecialCharTok{==} \ConstantTok{FALSE}
\NormalTok{     \}}
\NormalTok{   \}}
\NormalTok{   var3}
\NormalTok{\}}
\NormalTok{var2 }\OtherTok{\textless{}{-}} \DecValTok{1}\SpecialCharTok{:}\DecValTok{100}
\NormalTok{result }\OtherTok{=} \FunctionTok{getPrimeNumbers}\NormalTok{(var2)}
\FunctionTok{cat}\NormalTok{(}\StringTok{"Prime Numbers are: "}\NormalTok{,result)}
\end{Highlighting}
\end{Shaded}

\begin{verbatim}
## Prime Numbers are:  1 2 3 5 7 11 13 17 19 23 29 31 37 41 43 47 53 59 61 67 71 73 79 83 89 97
\end{verbatim}

\begin{Shaded}
\begin{Highlighting}[]
\CommentTok{\# Question 1 (b)}
\NormalTok{getithPrimeNumber }\OtherTok{\textless{}{-}} \ControlFlowTok{function}\NormalTok{(var,check)\{}
     \ControlFlowTok{while}\NormalTok{(}\ConstantTok{TRUE}\NormalTok{)\{}
\NormalTok{          primeNumbers }\OtherTok{=} \FunctionTok{getPrimeNumbers}\NormalTok{(}\FunctionTok{seq}\NormalTok{(}\DecValTok{1}\SpecialCharTok{:}\NormalTok{var))}
\NormalTok{          length }\OtherTok{=} \FunctionTok{length}\NormalTok{(primeNumbers)}
          \ControlFlowTok{if}\NormalTok{(length }\SpecialCharTok{==}\NormalTok{ check)\{}
               \ControlFlowTok{break}
\NormalTok{          \}}
\NormalTok{          var }\OtherTok{=}\NormalTok{ var }\SpecialCharTok{+} \DecValTok{1}
     
\NormalTok{     \}}
     \CommentTok{\# return }
\NormalTok{     primeNumbers[length]}
\NormalTok{\}}

\CommentTok{\# Use Custom Function}
\NormalTok{var }\OtherTok{=} \DecValTok{1}
\NormalTok{check }\OtherTok{=} \DecValTok{10}
\NormalTok{result }\OtherTok{=} \FunctionTok{getithPrimeNumber}\NormalTok{(var,check)}
\FunctionTok{cat}\NormalTok{(check,}\StringTok{"th Prime Number is:"}\NormalTok{,result,}\AttributeTok{sep =} \StringTok{""}\NormalTok{)}
\end{Highlighting}
\end{Shaded}

\begin{verbatim}
## 10th Prime Number is:23
\end{verbatim}

\begin{Shaded}
\begin{Highlighting}[]
\DocumentationTok{\#\#\#\# End solution \#\#\#\#}
\end{Highlighting}
\end{Shaded}

\hypertarget{question-2-1}{%
\subsubsection{Question 2}\label{question-2-1}}

\textbf{Create a matrix of size 3x3 called mat\_1:}\\
1. Iterate over all the values one by one and print the element as well
as the position in the matrix (row, col)

\begin{Shaded}
\begin{Highlighting}[]
\DocumentationTok{\#\#\#\# Start solution \#\#\#\#}
\end{Highlighting}
\end{Shaded}

\begin{Shaded}
\begin{Highlighting}[]
\CommentTok{\# Create Matrix 3x3}
\NormalTok{mat\_1 }\OtherTok{=} \FunctionTok{matrix}\NormalTok{(}\DecValTok{1}\SpecialCharTok{:}\DecValTok{9}\NormalTok{,}\DecValTok{3}\NormalTok{,}\DecValTok{3}\NormalTok{)}
\end{Highlighting}
\end{Shaded}

\begin{Shaded}
\begin{Highlighting}[]
\CommentTok{\# print matrix by elements}
\ControlFlowTok{for}\NormalTok{ (i }\ControlFlowTok{in} \FunctionTok{seq\_len}\NormalTok{(}\FunctionTok{nrow}\NormalTok{(mat\_1)))\{}
     \ControlFlowTok{for}\NormalTok{(j }\ControlFlowTok{in} \FunctionTok{seq\_len}\NormalTok{(}\FunctionTok{ncol}\NormalTok{(mat\_1)))\{}
          \FunctionTok{cat}\NormalTok{(}\StringTok{"Element at Position ("}\NormalTok{,i,}\StringTok{","}\NormalTok{,j,}\StringTok{") is :"}\NormalTok{,mat\_1[i,j],}\StringTok{"}\SpecialCharTok{\textbackslash{}n}\StringTok{"}\NormalTok{)}
\NormalTok{     \}}
\NormalTok{\}}
\end{Highlighting}
\end{Shaded}

\begin{verbatim}
## Element at Position ( 1 , 1 ) is : 1 
## Element at Position ( 1 , 2 ) is : 4 
## Element at Position ( 1 , 3 ) is : 7 
## Element at Position ( 2 , 1 ) is : 2 
## Element at Position ( 2 , 2 ) is : 5 
## Element at Position ( 2 , 3 ) is : 8 
## Element at Position ( 3 , 1 ) is : 3 
## Element at Position ( 3 , 2 ) is : 6 
## Element at Position ( 3 , 3 ) is : 9
\end{verbatim}

\begin{Shaded}
\begin{Highlighting}[]
\DocumentationTok{\#\#\#\# End solution \#\#\#\#}
\end{Highlighting}
\end{Shaded}

\hypertarget{good-job-you-have-successfully-completed-the-section-1}{%
\subsubsection{Good Job! You have successfully completed the
section!}\label{good-job-you-have-successfully-completed-the-section-1}}

\hypertarget{section-3-functions}{%
\subsection{Section 3: Functions}\label{section-3-functions}}

\hypertarget{question-1-2}{%
\subsubsection{Question 1}\label{question-1-2}}

\textbf{Create a function called gcd that finds the greatest common
divisor of two numbers a and b:}\\
1. a and b, should be taken as input.\\
2. The function must print the GCD as well as return it.\\
3. The output must be saved in a variable called answer.

\begin{Shaded}
\begin{Highlighting}[]
\DocumentationTok{\#\#\#\# Start solution \#\#\#\#}
\end{Highlighting}
\end{Shaded}

\begin{Shaded}
\begin{Highlighting}[]
\NormalTok{gcd }\OtherTok{\textless{}{-}} \ControlFlowTok{function}\NormalTok{(a,b)\{}
     \ControlFlowTok{if}\NormalTok{(a}\SpecialCharTok{\textless{}}\NormalTok{b) smaller }\OtherTok{=}\NormalTok{ a    }
     \ControlFlowTok{else}\NormalTok{ smaller }\OtherTok{=}\NormalTok{ b}

     \ControlFlowTok{for}\NormalTok{ (i }\ControlFlowTok{in} \DecValTok{1}\SpecialCharTok{:}\NormalTok{smaller) \{}
          \ControlFlowTok{if}\NormalTok{((a }\SpecialCharTok{\%\%}\NormalTok{ i }\SpecialCharTok{==} \DecValTok{0}\NormalTok{) }\SpecialCharTok{\&\&}\NormalTok{ (b }\SpecialCharTok{\%\%}\NormalTok{ i }\SpecialCharTok{==} \DecValTok{0}\NormalTok{))\{}
\NormalTok{               hcf }\OtherTok{=}\NormalTok{ i}
\NormalTok{          \}}
\NormalTok{     \}}
     
     \FunctionTok{cat}\NormalTok{(}\StringTok{"Greatest Common Divisor:"}\NormalTok{,hcf)}
     \FunctionTok{return}\NormalTok{(hcf)}
\NormalTok{\}}
\NormalTok{num1 }\OtherTok{=} \DecValTok{72}
\NormalTok{num2 }\OtherTok{=} \DecValTok{120}
\NormalTok{answer }\OtherTok{=} \FunctionTok{gcd}\NormalTok{(num2,num1)}
\end{Highlighting}
\end{Shaded}

\begin{verbatim}
## Greatest Common Divisor: 24
\end{verbatim}

\begin{Shaded}
\begin{Highlighting}[]
\DocumentationTok{\#\#\#\# End solution \#\#\#\#}
\end{Highlighting}
\end{Shaded}

\hypertarget{question-2-2}{%
\subsubsection{Question 2}\label{question-2-2}}

\textbf{Create a function called allConditionsMeet, that checks whether
two expressions evaluate to true:}\\
1. a and b, should be taken as input.\\
2. the function should check if a and b, both conditions, evaluate to
True.\\
3. The function must returns a boolean value.\\
4. Using a method, print the arguments that function takes.

\begin{Shaded}
\begin{Highlighting}[]
\DocumentationTok{\#\#\#\# Start solution \#\#\#\#}
\end{Highlighting}
\end{Shaded}

\begin{Shaded}
\begin{Highlighting}[]
\NormalTok{allConditionsMeet }\OtherTok{\textless{}{-}} \ControlFlowTok{function}\NormalTok{(a,b)\{}

     \ControlFlowTok{if}\NormalTok{((a }\SpecialCharTok{==} \ConstantTok{TRUE}\NormalTok{)}\SpecialCharTok{\&\&}\NormalTok{(b}\SpecialCharTok{==}\ConstantTok{TRUE}\NormalTok{))\{}
          \FunctionTok{return}\NormalTok{(}\ConstantTok{TRUE}\NormalTok{)}
\NormalTok{     \}}
\NormalTok{\}}
\NormalTok{exp1 }\OtherTok{=} \DecValTok{2} \SpecialCharTok{\textless{}} \DecValTok{5}
\NormalTok{exp2 }\OtherTok{=} \DecValTok{5} \SpecialCharTok{\textgreater{}=} \DecValTok{0}
\NormalTok{result }\OtherTok{=} \FunctionTok{allConditionsMeet}\NormalTok{(exp1,exp2)}
\FunctionTok{cat}\NormalTok{(}\StringTok{"Result is:"}\NormalTok{,result,}\StringTok{"}\SpecialCharTok{\textbackslash{}n}\StringTok{"}\NormalTok{)}
\end{Highlighting}
\end{Shaded}

\begin{verbatim}
## Result is: TRUE
\end{verbatim}

\begin{Shaded}
\begin{Highlighting}[]
\FunctionTok{print}\NormalTok{(}\StringTok{"Arguments:"}\NormalTok{)}
\end{Highlighting}
\end{Shaded}

\begin{verbatim}
## [1] "Arguments:"
\end{verbatim}

\begin{Shaded}
\begin{Highlighting}[]
\FunctionTok{print}\NormalTok{(}\FunctionTok{formals}\NormalTok{(allConditionsMeet))}
\end{Highlighting}
\end{Shaded}

\begin{verbatim}
## $a
## 
## 
## $b
\end{verbatim}

\begin{Shaded}
\begin{Highlighting}[]
\DocumentationTok{\#\#\#\# End solution \#\#\#\#}
\end{Highlighting}
\end{Shaded}

\hypertarget{good-job-you-have-successfully-completed-the-section-2}{%
\subsubsection{Good Job! You have successfully completed the
section!}\label{good-job-you-have-successfully-completed-the-section-2}}

\hypertarget{section-4-vectorized-operations}{%
\subsection{Section 4: Vectorized
Operations}\label{section-4-vectorized-operations}}

\hypertarget{question-1-3}{%
\subsubsection{Question 1}\label{question-1-3}}

\textbf{Create two matrices matrix\_1 and matrix\_2 of dimensions 2x3
and 3x2:}\\
1. Perform element-wise multiplication.\\
2. Perform matrix multipilcation.

\textbf{Create a 2x2 matrix my\_mat:}\\
1. Write a function to find the determinant of the matrix.

\begin{Shaded}
\begin{Highlighting}[]
\DocumentationTok{\#\#\#\# Start solution \#\#\#\#}
\end{Highlighting}
\end{Shaded}

\begin{Shaded}
\begin{Highlighting}[]
\CommentTok{\#Create Matrices}
\NormalTok{matrix\_1 }\OtherTok{\textless{}{-}} \FunctionTok{matrix}\NormalTok{(}\DecValTok{1}\SpecialCharTok{:}\DecValTok{6}\NormalTok{,}\DecValTok{2}\NormalTok{,}\DecValTok{3}\NormalTok{)}
\NormalTok{matrix\_2 }\OtherTok{\textless{}{-}} \FunctionTok{matrix}\NormalTok{(}\DecValTok{1}\SpecialCharTok{:}\DecValTok{6}\NormalTok{,}\DecValTok{3}\NormalTok{,}\DecValTok{2}\NormalTok{)}
\NormalTok{matrix\_1}
\end{Highlighting}
\end{Shaded}

\begin{verbatim}
##      [,1] [,2] [,3]
## [1,]    1    3    5
## [2,]    2    4    6
\end{verbatim}

\begin{Shaded}
\begin{Highlighting}[]
\NormalTok{matrix\_2}
\end{Highlighting}
\end{Shaded}

\begin{verbatim}
##      [,1] [,2]
## [1,]    1    4
## [2,]    2    5
## [3,]    3    6
\end{verbatim}

\begin{Shaded}
\begin{Highlighting}[]
\CommentTok{\# element{-}wise multiplication}
\CommentTok{\#elements = c()}
\CommentTok{\#for (i in seq\_len(nrow(matrix\_1)))\{}
\CommentTok{\#     a = c(matrix\_1[i,]) * c(matrix\_2[i,])}
\CommentTok{\#     elements \textless{}{-} c(elements,a)}
\CommentTok{\#\}}
\CommentTok{\#elementWiseMultiplication = matrix(elements,2,3)}
\CommentTok{\#elementWiseMultiplication}

\CommentTok{\# approach 2:}
\NormalTok{element\_mat }\OtherTok{=} \FunctionTok{c}\NormalTok{(matrix\_1) }\SpecialCharTok{*} \FunctionTok{c}\NormalTok{(matrix\_2)}
\FunctionTok{print}\NormalTok{(}\FunctionTok{matrix}\NormalTok{(element\_mat,}\DecValTok{2}\NormalTok{,}\DecValTok{3}\NormalTok{))}
\end{Highlighting}
\end{Shaded}

\begin{verbatim}
##      [,1] [,2] [,3]
## [1,]    1    9   25
## [2,]    4   16   36
\end{verbatim}

\begin{Shaded}
\begin{Highlighting}[]
\CommentTok{\# approach 3:}
\CommentTok{\#print("Element Wise Multiplication: ",matrix\_1 * matrix\_2)}
\end{Highlighting}
\end{Shaded}

\begin{Shaded}
\begin{Highlighting}[]
\CommentTok{\# matrix multiplication}
\NormalTok{matrixMultiplication }\OtherTok{=}\NormalTok{ matrix\_1 }\SpecialCharTok{\%*\%}\NormalTok{ matrix\_2}
\NormalTok{matrixMultiplication}
\end{Highlighting}
\end{Shaded}

\begin{verbatim}
##      [,1] [,2]
## [1,]   22   49
## [2,]   28   64
\end{verbatim}

\begin{Shaded}
\begin{Highlighting}[]
\CommentTok{\# determinant of matrix}
\NormalTok{matrix\_3 }\OtherTok{=} \FunctionTok{matrix}\NormalTok{(}\DecValTok{5}\SpecialCharTok{:}\DecValTok{8}\NormalTok{,}\DecValTok{2}\NormalTok{,}\DecValTok{2}\NormalTok{)}
\NormalTok{matrix\_3}
\end{Highlighting}
\end{Shaded}

\begin{verbatim}
##      [,1] [,2]
## [1,]    5    7
## [2,]    6    8
\end{verbatim}

\begin{Shaded}
\begin{Highlighting}[]
\FunctionTok{print}\NormalTok{(}\FunctionTok{det}\NormalTok{(matrix\_3))}
\end{Highlighting}
\end{Shaded}

\begin{verbatim}
## [1] -2
\end{verbatim}

\begin{Shaded}
\begin{Highlighting}[]
\DocumentationTok{\#\#\#\# End solution \#\#\#\#}
\end{Highlighting}
\end{Shaded}

\hypertarget{good-job-you-have-successfully-completed-the-section-3}{%
\subsubsection{Good Job! You have successfully completed the
section!}\label{good-job-you-have-successfully-completed-the-section-3}}

\hypertarget{section-5-date-and-time-in-r}{%
\subsection{Section 5: Date and Time in
R}\label{section-5-date-and-time-in-r}}

\hypertarget{question-1-4}{%
\subsubsection{Question 1}\label{question-1-4}}

\textbf{Use the current date and time and store them into variables
curr\_date and curr\_time respectively: {[}use sys{]}}\\
1. Convert both into date and time objects using the appropriate
functions.\\
2. Print the weekday, year, second and hour using the appropriate
function and variables.

\begin{Shaded}
\begin{Highlighting}[]
\DocumentationTok{\#\#\#\# Start solution \#\#\#\#}
\end{Highlighting}
\end{Shaded}

\begin{Shaded}
\begin{Highlighting}[]
\CommentTok{\# get date \& time}
\NormalTok{curr\_time }\OtherTok{=} \FunctionTok{Sys.time}\NormalTok{()}
\NormalTok{curr\_time}
\end{Highlighting}
\end{Shaded}

\begin{verbatim}
## [1] "2022-09-07 06:25:13 PDT"
\end{verbatim}

\begin{Shaded}
\begin{Highlighting}[]
\NormalTok{curr\_date }\OtherTok{=} \FunctionTok{as.Date}\NormalTok{(curr\_time)}
\NormalTok{curr\_date }
\end{Highlighting}
\end{Shaded}

\begin{verbatim}
## [1] "2022-09-07"
\end{verbatim}

\begin{Shaded}
\begin{Highlighting}[]
\CommentTok{\# get Objects}
\NormalTok{datetObject }\OtherTok{=} \FunctionTok{as.POSIXct}\NormalTok{(curr\_date)}
\NormalTok{datetObject}
\end{Highlighting}
\end{Shaded}

\begin{verbatim}
## [1] "2022-09-06 17:00:00 PDT"
\end{verbatim}

\begin{Shaded}
\begin{Highlighting}[]
\NormalTok{timeObject }\OtherTok{=} \FunctionTok{as.POSIXlt}\NormalTok{(curr\_time)}
\FunctionTok{names}\NormalTok{(}\FunctionTok{unclass}\NormalTok{(timeObject))}
\end{Highlighting}
\end{Shaded}

\begin{verbatim}
##  [1] "sec"    "min"    "hour"   "mday"   "mon"    "year"   "wday"   "yday"  
##  [9] "isdst"  "zone"   "gmtoff"
\end{verbatim}

\begin{Shaded}
\begin{Highlighting}[]
\CommentTok{\#get attributes}
\FunctionTok{print}\NormalTok{(timeObject}\SpecialCharTok{$}\NormalTok{wday)}
\end{Highlighting}
\end{Shaded}

\begin{verbatim}
## [1] 3
\end{verbatim}

\begin{Shaded}
\begin{Highlighting}[]
\FunctionTok{print}\NormalTok{(timeObject}\SpecialCharTok{$}\NormalTok{year)}
\end{Highlighting}
\end{Shaded}

\begin{verbatim}
## [1] 122
\end{verbatim}

\begin{Shaded}
\begin{Highlighting}[]
\FunctionTok{print}\NormalTok{(timeObject}\SpecialCharTok{$}\NormalTok{sec)}
\end{Highlighting}
\end{Shaded}

\begin{verbatim}
## [1] 13.98435
\end{verbatim}

\begin{Shaded}
\begin{Highlighting}[]
\FunctionTok{print}\NormalTok{(timeObject}\SpecialCharTok{$}\NormalTok{hour)}
\end{Highlighting}
\end{Shaded}

\begin{verbatim}
## [1] 6
\end{verbatim}

\begin{Shaded}
\begin{Highlighting}[]
\DocumentationTok{\#\#\#\# End solution \#\#\#\#}
\end{Highlighting}
\end{Shaded}

\hypertarget{question-2-3}{%
\subsubsection{Question 2}\label{question-2-3}}

\textbf{Create a variable to store current date/time}\\
1. Create another variable that stores and set the timezone as GMT-5\\
2. Find the difference between your current time and the GMT-5 timezone.

\begin{Shaded}
\begin{Highlighting}[]
\DocumentationTok{\#\#\#\# Start solution \#\#\#\#}
\end{Highlighting}
\end{Shaded}

\begin{Shaded}
\begin{Highlighting}[]
\NormalTok{dateWithTimeZOne }\OtherTok{=} \FunctionTok{as.POSIXlt}\NormalTok{(}\FunctionTok{Sys.time}\NormalTok{(),}\AttributeTok{tz=}\StringTok{"GMT{-}5"}\NormalTok{)}
\end{Highlighting}
\end{Shaded}

\begin{verbatim}
## Warning in as.POSIXlt.POSIXct(Sys.time(), tz = "GMT-5"): unknown timezone
## 'GMT-5'
\end{verbatim}

\begin{Shaded}
\begin{Highlighting}[]
\CommentTok{\# Print time for both time zones}
\NormalTok{timeObject}
\end{Highlighting}
\end{Shaded}

\begin{verbatim}
## [1] "2022-09-07 06:25:13 PDT"
\end{verbatim}

\begin{Shaded}
\begin{Highlighting}[]
\NormalTok{dateWithTimeZOne}
\end{Highlighting}
\end{Shaded}

\begin{verbatim}
## [1] "2022-09-07 18:25:14 GMT"
\end{verbatim}

\begin{Shaded}
\begin{Highlighting}[]
\CommentTok{\# difference}
\NormalTok{difference }\OtherTok{=}\NormalTok{ timeObject }\SpecialCharTok{{-}}\NormalTok{ dateWithTimeZOne}
\end{Highlighting}
\end{Shaded}

\begin{verbatim}
## Warning in as.POSIXct.POSIXlt(time2): unknown timezone 'GMT-5'
\end{verbatim}

\begin{Shaded}
\begin{Highlighting}[]
\NormalTok{difference}
\end{Highlighting}
\end{Shaded}

\begin{verbatim}
## Time difference of -0.120754 secs
\end{verbatim}

\begin{Shaded}
\begin{Highlighting}[]
\DocumentationTok{\#\#\#\# End solution \#\#\#\#}
\end{Highlighting}
\end{Shaded}

\hypertarget{good-job-you-have-successfully-completed-the-section-4}{%
\subsubsection{Good Job! You have successfully completed the
section!}\label{good-job-you-have-successfully-completed-the-section-4}}

\hypertarget{section-6-loop-functions}{%
\subsection{Section 6: Loop Functions}\label{section-6-loop-functions}}

\hypertarget{question-1-5}{%
\subsubsection{Question 1}\label{question-1-5}}

\textbf{Create a function to calculate mean and standard deviation of
the provided data} 1. Create a sequence of numbers from 100 to 150 store
in a variable called numbers. 1. Use lapply, sapply, apply and tapply to
implement the functions on ``numbers'' {[}only on the second half of the
sequence for tapply{]}

\begin{Shaded}
\begin{Highlighting}[]
\DocumentationTok{\#\#\#\# Start solution \#\#\#\#}
\end{Highlighting}
\end{Shaded}

\begin{Shaded}
\begin{Highlighting}[]
\NormalTok{calculateMSD }\OtherTok{\textless{}{-}} \ControlFlowTok{function}\NormalTok{(data) }\FunctionTok{c}\NormalTok{(}\AttributeTok{mean=}\FunctionTok{mean}\NormalTok{(data),}\AttributeTok{standard\_deviation =}\FunctionTok{sd}\NormalTok{(data))}
\CommentTok{\# data}
\NormalTok{numbers }\OtherTok{=} \DecValTok{100}\SpecialCharTok{:}\DecValTok{150}

\CommentTok{\# lapply returned list}
\FunctionTok{lapply}\NormalTok{(}\FunctionTok{list}\NormalTok{(numbers),calculateMSD)}
\end{Highlighting}
\end{Shaded}

\begin{verbatim}
## [[1]]
##               mean standard_deviation 
##          125.00000           14.86607
\end{verbatim}

\begin{Shaded}
\begin{Highlighting}[]
\CommentTok{\#sapply returned data (matrix,array)}
\FunctionTok{sapply}\NormalTok{(}\FunctionTok{list}\NormalTok{(numbers),calculateMSD)}
\end{Highlighting}
\end{Shaded}

\begin{verbatim}
##                         [,1]
## mean               125.00000
## standard_deviation  14.86607
\end{verbatim}

\begin{Shaded}
\begin{Highlighting}[]
\CommentTok{\# apply work on array or matrix, returned (matrix,array)}
\FunctionTok{apply}\NormalTok{(}\FunctionTok{matrix}\NormalTok{(numbers,}\DecValTok{1}\NormalTok{,}\DecValTok{51}\NormalTok{),}\AttributeTok{MARGIN =}  \DecValTok{1}\NormalTok{,}\AttributeTok{FUN =}\NormalTok{ calculateMSD)}
\end{Highlighting}
\end{Shaded}

\begin{verbatim}
##                         [,1]
## mean               125.00000
## standard_deviation  14.86607
\end{verbatim}

\begin{Shaded}
\begin{Highlighting}[]
\CommentTok{\# tapply works on ragged array {-}{-} vector returned array}
\CommentTok{\# create a factor of same length as of data : for indexing}
\NormalTok{f }\OtherTok{=} \FunctionTok{gl}\NormalTok{(}\DecValTok{5}\NormalTok{,}\DecValTok{5}\NormalTok{)}
\FunctionTok{tapply}\NormalTok{(numbers[}\DecValTok{26}\SpecialCharTok{:}\DecValTok{50}\NormalTok{], f, calculateMSD)}
\end{Highlighting}
\end{Shaded}

\begin{verbatim}
## $`1`
##               mean standard_deviation 
##         127.000000           1.581139 
## 
## $`2`
##               mean standard_deviation 
##         132.000000           1.581139 
## 
## $`3`
##               mean standard_deviation 
##         137.000000           1.581139 
## 
## $`4`
##               mean standard_deviation 
##         142.000000           1.581139 
## 
## $`5`
##               mean standard_deviation 
##         147.000000           1.581139
\end{verbatim}

\begin{Shaded}
\begin{Highlighting}[]
\CommentTok{\# printed mean with indexing of 5 and standard deviation with levels 5}
\end{Highlighting}
\end{Shaded}

\begin{Shaded}
\begin{Highlighting}[]
\DocumentationTok{\#\#\#\# End solution \#\#\#\#}
\end{Highlighting}
\end{Shaded}

\hypertarget{question-2-4}{%
\subsubsection{Question 2}\label{question-2-4}}

\textbf{Create a matrix of dimensions 4x4} 1. Find the row-wise and
column-wise mean of the matrix. 2. Print the values.

\begin{Shaded}
\begin{Highlighting}[]
\DocumentationTok{\#\#\#\# Start solution \#\#\#\#}
\end{Highlighting}
\end{Shaded}

\begin{Shaded}
\begin{Highlighting}[]
\NormalTok{matrix\_4 }\OtherTok{\textless{}{-}} \FunctionTok{matrix}\NormalTok{(}\DecValTok{21}\SpecialCharTok{:}\DecValTok{36}\NormalTok{,}\DecValTok{4}\NormalTok{,}\DecValTok{4}\NormalTok{)}
\NormalTok{matrix\_4}
\end{Highlighting}
\end{Shaded}

\begin{verbatim}
##      [,1] [,2] [,3] [,4]
## [1,]   21   25   29   33
## [2,]   22   26   30   34
## [3,]   23   27   31   35
## [4,]   24   28   32   36
\end{verbatim}

\begin{Shaded}
\begin{Highlighting}[]
\NormalTok{rm }\OtherTok{=} \FunctionTok{rowMeans}\NormalTok{(matrix\_4)}
\FunctionTok{cat}\NormalTok{(}\StringTok{"Row Means:"}\NormalTok{,rm)}
\end{Highlighting}
\end{Shaded}

\begin{verbatim}
## Row Means: 27 28 29 30
\end{verbatim}

\begin{Shaded}
\begin{Highlighting}[]
\NormalTok{cm }\OtherTok{=} \FunctionTok{colMeans}\NormalTok{(matrix\_4)}
\FunctionTok{cat}\NormalTok{(}\StringTok{"Column Means:"}\NormalTok{,cm)}
\end{Highlighting}
\end{Shaded}

\begin{verbatim}
## Column Means: 22.5 26.5 30.5 34.5
\end{verbatim}

\begin{Shaded}
\begin{Highlighting}[]
\DocumentationTok{\#\#\#\# End solution \#\#\#\#}
\end{Highlighting}
\end{Shaded}

\hypertarget{good-job-you-have-successfully-completed-the-section-5}{%
\subsubsection{Good Job! You have successfully completed the
section!}\label{good-job-you-have-successfully-completed-the-section-5}}

\hypertarget{section-7-data-split}{%
\subsection{Section 7: Data Split}\label{section-7-data-split}}

\hypertarget{question-1-6}{%
\subsubsection{Question 1}\label{question-1-6}}

\textbf{Using the data frame Orange:}\\
1. Using split function to break down the dataset on circumference and
store it in `split\_data' variable.\\
2. Print the values for split\_data where circumference is 30 and 75.\\
3. Find the average age of the tree when the circumference is 30 and
when circumference is 214.

The dataset is loaded and the variable Orange contains the respective
dataset.

\begin{Shaded}
\begin{Highlighting}[]
\FunctionTok{library}\NormalTok{(datasets)}
\end{Highlighting}
\end{Shaded}

\begin{Shaded}
\begin{Highlighting}[]
\FunctionTok{head}\NormalTok{(Orange)}
\end{Highlighting}
\end{Shaded}

\begin{verbatim}
##   Tree  age circumference
## 1    1  118            30
## 2    1  484            58
## 3    1  664            87
## 4    1 1004           115
## 5    1 1231           120
## 6    1 1372           142
\end{verbatim}

\begin{Shaded}
\begin{Highlighting}[]
\DocumentationTok{\#\#\#\# Start solution \#\#\#\#}
\end{Highlighting}
\end{Shaded}

\begin{Shaded}
\begin{Highlighting}[]
\NormalTok{split\_data }\OtherTok{=} \FunctionTok{split}\NormalTok{(Orange,Orange}\SpecialCharTok{$}\NormalTok{circumference)}
\CommentTok{\#class(split\_data)}
\CommentTok{\#names(unclass(split\_data))}

\CommentTok{\# Data for 30}
\NormalTok{split\_data}\SpecialCharTok{$}\StringTok{\textasciigrave{}}\AttributeTok{30}\StringTok{\textasciigrave{}}
\end{Highlighting}
\end{Shaded}

\begin{verbatim}
##    Tree age circumference
## 1     1 118            30
## 15    3 118            30
## 29    5 118            30
\end{verbatim}

\begin{Shaded}
\begin{Highlighting}[]
\CommentTok{\# Data for 75}
\NormalTok{split\_data}\SpecialCharTok{$}\StringTok{\textasciigrave{}}\AttributeTok{75}\StringTok{\textasciigrave{}}
\end{Highlighting}
\end{Shaded}

\begin{verbatim}
##    Tree age circumference
## 17    3 664            75
\end{verbatim}

\begin{Shaded}
\begin{Highlighting}[]
\CommentTok{\#average age for 30}
\NormalTok{data1 }\OtherTok{\textless{}{-}}\NormalTok{ split\_data}\SpecialCharTok{$}\StringTok{\textasciigrave{}}\AttributeTok{30}\StringTok{\textasciigrave{}}
\NormalTok{dataWithAge1 }\OtherTok{=} \FunctionTok{list}\NormalTok{(data1}\SpecialCharTok{$}\NormalTok{age)}

\NormalTok{age1Mean }\OtherTok{=}\FunctionTok{sapply}\NormalTok{(dataWithAge1, mean)}
\FunctionTok{cat}\NormalTok{(}\StringTok{"Average age for 30:"}\NormalTok{,age1Mean)}
\end{Highlighting}
\end{Shaded}

\begin{verbatim}
## Average age for 30: 118
\end{verbatim}

\begin{Shaded}
\begin{Highlighting}[]
\CommentTok{\# average age for 214}
\NormalTok{data2 }\OtherTok{\textless{}{-}}\NormalTok{ split\_data}\SpecialCharTok{$}\StringTok{\textasciigrave{}}\AttributeTok{214}\StringTok{\textasciigrave{}}
\NormalTok{dataWithAge2 }\OtherTok{=} \FunctionTok{list}\NormalTok{(data2}\SpecialCharTok{$}\NormalTok{age)}

\NormalTok{age2Mean }\OtherTok{=}\FunctionTok{sapply}\NormalTok{(dataWithAge2, mean)}
\FunctionTok{cat}\NormalTok{(}\StringTok{"Average age for 214:"}\NormalTok{,age2Mean)}
\end{Highlighting}
\end{Shaded}

\begin{verbatim}
## Average age for 214: 1582
\end{verbatim}

\begin{Shaded}
\begin{Highlighting}[]
\DocumentationTok{\#\#\#\# End solution \#\#\#\#}
\end{Highlighting}
\end{Shaded}

\hypertarget{save-it-and-push-.rmd-and-.html-file-to-your-github-repository}{%
\subsubsection{Save it and push .Rmd and .html file to your Github
Repository}\label{save-it-and-push-.rmd-and-.html-file-to-your-github-repository}}

\hypertarget{good-job-you-have-successfully-finished-the-course}{%
\subsection{GOOD JOB! You have successfully finished the
course!}\label{good-job-you-have-successfully-finished-the-course}}

\end{document}
